%! TEX root = ../main.tex
\documentclass[main]{subfiles}

\begin{document}
% \begin{titlepage}

% \vol{34} % 号
% \year{2015} % 出版年
% \pages{15}{18} % 開始ページ・終了ページ
% \received{1 January 20xx} % 原稿受理日
% \revised{1 March, 20xx} % 修正原稿受理日
% \accepted{1 April, 20xx} % 論文受理日

\date{2023/04/08}
\doc{DRR-xxxxx}


%----- タイトルページ日本語部分の設定 -------------------------------
% \hz{#1}を全角空白の#1倍だけの空白を空けるコマンドとして定義しているので
% 著者名の文字間隔を調整したい際に利用できる.  
% もちろん全角空白や ~ も同様に利用可能.

\jtitle{「ながれ」原稿執筆要項 (\LaTeX)} % 日本語タイトル

\headertitle{「ながれ」原稿執筆要項}% ヘッダ用タイトル
%                     ↑↑ヘッダ用タイトルが長い場合は省略形で書くこと
\headerauthor{流体太郎・力学次郎・Fluid MECHANICS}% ヘッダ用著者名

%----- 第1著者(通常はこれを使う) -----
% \email{jsfm@nagare.or.jp} % Emailアドレス
\jaffil{流体力学大学 理学部} % 著者所属
\jauthor{流 体 太 郎} % 著者名


%----- 第2著者(通常) -----
% \jaffil{株式会社ながれ 力学研究所}
% \jauthor{力 学 次 郎}

%----- 第3著者(現所属を追記する場合)-----
% \jaffil{$^{\dagger\dagger}$Faculty of Engineering, University of Fluid}
% \jpresentaffili{$^{\dagger\dagger}$Present affiliation: Ryutai Engineering, Co. Ltd.}
% \jauthor{Fluid MECHANICS}

%----- 第4著者(通常) -----
% \jaffil{流体大学 工学部}
% \jauthor{な が れ 華 子}



%----- タイトルページ英語部分の設定 ---------------------------------
% 現所属を追記する場合は,記号も書く.詳細は,3.2参照.

% \title{Instructions for Journal of Japan Society of Fluid Mechanics\\
%     (for Electronically Submitted Manuscripts)} % 英語タイトル

% \affil{$^{**}$Faculty of Science, University of Fluid Mechanics} %英語所属
% \author{*Taro Ryutai}  % 英語著者名

% \affil{Research Center of Mechanics, Nagare Co., Ltd.}
% \author{Jiro Rikigaku}

% \affil{$^{\dagger\dagger}$Faculty of Engineering, University of Fluid}
% % \presentaffili{$^{\dagger\dagger}$Present affiliation: Ryutai Engineering, Co. Ltd.}
% \author{Fluid Mechanics}

% \affil{Faculty of Engineering, University of Fluid Mechanics}
% \author{Hanako Nagare}


% \keywords{fluid mechanics, instructions, page format キーワード} %キーワード
\keywords{} %キーワード

\abst{%
    Abstract of your paper should be written within 200 words. %
    ・ ・ ・ ・ ・ ・ ・ ・ ・ ・ ・ ・ ・ ・ ・ ・ ・ ・ ・ ・ ・ ・ ・ %
    ・ ・ ・ ・ ・ ・ ・ ・ ・ ・ ・ ・ ・ ・ ・ ・ ・ ・ ・ ・ ・ ・ ・ %
    ・ ・ ・ ・ ・ ・ ・ ・ ・ ・ ・ ・ ・ ・ ・ ・ ・ ・ ・ ・ ・ ・ ・ %
    ・ ・ ・ ・ ・ ・ ・ ・ ・ ・ ・ ・ ・ ・ ・ ・ ・ ・ ・ ・ ・ ・ ・ %
    ・ ・ ・ ・ ・ ・ ・ ・ ・ ・ ・ ・ ・ ・ ・ ・ ・ ・ ・ ・ ・ ・ ・ %
    ・ ・ ・ ・ ・ ・ ・ ・ ・ ・ ・ ・ ・ ・ ・ ・ ・ ・ ・ ・ ・ ・ ・ }





\maketitle

% \end{titlepage}
\end{document}
