%! TEX root = ../main.tex
\documentclass[main]{subfiles}

\begin{document}

    \section{はじめに}
    この執筆要項は,日本流体力学会誌「ながれ」の原稿テンプレートです.
    原稿は,この執筆要項に従って作成してください.
    原稿はA4版のPDFで投稿することとし,それを版下原稿とします.
    言語は日本語あるいは英語とします.以下,余白やフォントの指定を記しますが,
    {\LaTeX}の場合はクラスファイル”nagare.cls”が正しく動作する限り自動的に調整されます.
    動作は「TeX Live2017, 2018, 2019, 2020」,OS:CentOS7, Ubuntu 18.04,Windows7 64bit,Windows10 64bit,MAC OS で確認しています.
    
    \section{原稿用紙}
    原稿用紙の用紙サイズはA4サイズとします.
    上下左右のマージンは20mmとして下さい.
    1ページ目は,邦文題目,著者所属,著者名,邦文要旨,英文題目,英文著者名,代表著者E-mailアドレスを1段組で記し,その下より2段組で本文を書き始めて下さい.
    各セクションの行間値は3章で規定していますが,基本的にフォントサイズの1.5倍としています.
    
    \section{論文体裁}   

    \subsection{邦文題目}   
        18ポイント明朝体(英数字はTimes)あるいはこれに準じるフォントを用いてセンタリングして下さい.
        複数行にわたる場合の行間は24ポイントとして下さい.
        LaTeXの場合は \jtitle コマンドを用いて指定してください.
        複数行にわたる場合は \\ で分割できます.
    
    \subsection{邦文著者名}    
        邦文題目との間を1行程度空け,
        9ポイントで著者所属,14ポイントで著者名を右寄せで記して下さい.
        複数著者の場合の行間は20ポイントとして下さい.
        代表著者名の右側には記号($\ast$)を付し,英文著者名の下にE-mailアドレスを明記して下さい.
        現在の所属を別載する場合は,和文(国内)または英文(海外)の組織名を記載してください.
        代表著者の記号は${**}$を,代表著者以外については,順に${\dagger\dagger}$,${\ddagger\ddagger}$,${\S\S}$を用いてください.
        % LaTeX の場合,\jpresentaffili を使用してください.記号は見本に参考に執筆者で入力してください.
    
    \subsection{英文題目}
    
        16ポイントTimes体あるいはこれに準じるフォントを用いてセンタリングし,
        Main Wordsの最初の文字のみを大文字にして下さい.
        複数行にわたる場合の行間は24ポイントとして下さい.
    
    \subsection{英文著者名・英文所属機関名}
    
        英文題目との間を1行程度空け,
        14ポイントTimes体あるいはこれに準じるフォントで
        1行目に英文著者名を書き,
        その右に9ポイントTimes体あるいはこれに準じるフォントで
        英文所属機関名を書いて下さい.
        複数著者の場合の行間は20ポイントとして下さい.
    
    \subsection{英文要旨(特集・小特集,連載は不要)}
    
        英文著者名・英文所属機関名
        (最終原稿の場合は著者原稿受理日等)の間を1行空け,
        200words以内の英文要旨を書いて下さい.
        この際,9ポイントのTimes体あるいはこれに準じるフォントを用い,
        行間は14ポイントとして下さい.
        また,英文要旨の後に,キーワードを見本と同様の形式で記入して下さい.
        英文要旨部分のマージンは,本文より左右10mmずつ多くとって下さい.
    
    \subsection{本文}
    
        1ページ目は英文要旨(特集,解説の場合はE-mail)との間を1行程度空けて書き始め,
        2ページ目以降は原稿用紙上端から記して下さい.本文は2段組で,段と段の間隔は5mmとして下さい.
        10ポイント明朝体(英数字はTimes)あるいはこれに準じるフォントを用い,行間は15ポイントとし
        ます.文字間隔は標準の設定とします.1行当りの文字数は25文字,1ペ
        ージの行数は48行程度となります.
        句読点は和文コンマおよび和文ピリオドを用いて下さい.読点(,)や句点(.)ではありません.
    
        \subsubsection{見出し(見出しが1行以上になる場合は%
                この例のようにインデントして折り返す。
                繰り返すが見出しが1行以上になる場合はこの例のようにインデントして折り返す。
                やはり繰り返すが見出しが1行以上になる場合はこの例のようにインデントして折り返す。)}
            もちろんLaTeXコンパイラが勝手にインデントしてくれるので
            原稿ではインデントする必要はない.
    
            第一レベル(章)の見出しの前後はそれぞれ1行空けて下さい.
            第二レベル(節),第三レベル(項)の見出しの前には1行の空行をとり,
            見出しの後には空行を取らずに本文を書いて下さい.
            章,節,項の見出しともに,
            見出し番号は10ポイントTimesあるいはこれに準じるフォントの
            \textbf{Bold}体を用い,
            見出しの文字には\textgt{ゴシック体}(英数字はTimes)
            あるいはこれに準じるフォントを用いて下さい.
    
    \subsection{図表}
    
        図表はそれぞれを記述した本文とできるだけ離れないような位置に配置して下さい.
        また,図表内の文字は本文と同程度の大きさとして下さい.
        図や写真を張り込む場合は,
        そのための空白スペースを設け,
        張り込まれた後の配置と対応するようにキャプションを置いて下さい.
    
        図表の横幅は「2段ぶち抜き(170mm)」あるいは「1段の幅(80mm)」の
        いずれかとして下さい.
        図表と文書本体の間には1〜2行程度の空白を空けて区別を明確にして下さい.
        図表のキャプションは邦文あるいは英文で9ポイントとし,
        図の場合は図の下部に,表の場合は表の上部に配置して下さい.
        表記はそれぞれ,図1,図2\ldots(またはFig. 1,Fig. 2\ldots),
        表1,表2\ldots(またはTable 1,Table 2\ldots)とし,
        本文中でも同様に記載して下さい.
    
        \begin{table}[t]
            \caption{表のキャプションは表の上に置く.
                このように長いときはインデントして折り返す.このように長いときはインデントして折り返す.このように長いときはインデントして折り返す.このように長いときはインデントして折り返す.}
            \label{TableSample}
            \begin{center}
                \begin{tabular}{c|c|c}\hline\hline
                    資料番号 & 高さ $h$ (m) & 幅 $w$ (m) \\\hline
                    1    & 1.5        & 2.1       \\
                    2    & 1.2        & 3.2       \\
                    3    & 1.1        & 2.6       \\\hline\hline
                \end{tabular}
            \end{center}
        \end{table}
    
        \begin{table*}[t]
            \caption{光速度の測定の歴史}
            \label{table:SpeedOfLight}
            \begin{center}
                \begin{tabular}{clll}\hline
                    西暦   & 測定者      & 測定方法               & 測定結果                  \\
                         &          &                    & $\times 10^8$ [m/sec] \\
                    \hline \hline
                    1638 & Galileo  & 二人が離れてランプの光を見る     & (音速10倍以上)             \\
                    1675 & Roemer   & 木星の衛星の観測から         & 2                     \\
                    1728 & Bradley  & 星の収差から             & 3.01                  \\
                    1849 & Fizeau   & 高速に回転する歯車を通過する光を見る & 3.133                 \\
                    1862 & Foucault & 高速に回転する鏡の光の角度変化    & 2.99796               \\
                    今日   & (定義)     &                    & 2.99792458            \\
                    \hline
                \end{tabular}
            \end{center}
        \end{table*}
    
        \begin{figure}[hb]
            \begin{center}
                \includegraphics[width=6.5cm]{fig}
            \end{center}
            \caption{小さな図の例(図のキャプションは図の下に置く)}%
            \label{小さな図}
        \end{figure}
    
        \begin{figure*}[t]
            \begin{center}
                \includegraphics[width=14cm]{fig}
            \end{center}
            \caption{大きな図の例(図のキャプションは図の下に置く)}%
            \label{大きな図}
        \end{figure*}
    
        カラーで作成された図であっても,
        著者からの特別の希望がない場合は白黒で印刷されます
        (オンライン版はカラーのまま掲載されます).
        白黒で出力された場合の品質に注意して作成して下さい.
        また,カラーでの出力を希望する場合は,
        別途カラー用の印刷費用がかかります
        (詳細は投稿の手引き5節を参照).
    
    \subsection{数式および数学記号}
    
        数式は,
        \begin{equation}
            G=\sum^{\infty}_{n=0} b_n(t)
            \label{eq1}
        \end{equation}
        のように,本文と独立している場合は,本文より1文字分字下げして配置し,
        式番号は括弧書きで右詰めに配置して下さい.数式は,本文と独立している場合でも,
        本文中の場合でも,同じ数式用のフォントを用いて作成して下さい.
        本文中で引用する場合には,式(\ref{eq1})(または Eq. (\ref{eq1}))と記載して下さい.
    
        \begin{align}
            &&\mbox{d}\left\{\sum^N_{i=1}\frac{1}{2}m_i\left[\left(\frac{\mbox{d}x_i}{\mbox{d}t}\right)^2+\left(\frac{\mbox{d}y_i}{\mbox{d}t}\right)^2+\left(\frac{\mbox{d}z_i}{\mbox{d}t}\right)^2+\right]\right\}=\sum^N_{i=1}(X_i\mbox{d}x_i+Y_i\mbox{d}y_i+Z_i\mbox{d}z_i)\\
            &&\bar{C}(t)=\frac{1}{N}\sum^N_{i=1}C_i(t)\\
            &&\frac{p_v-p_{sat}}{p_{sat}}=-\left(2.13204+2\sqrt{\pi}\frac{1-\zeta}{\zeta}\right)\frac{(u_v-u_{int})\cdot n}{\sqrt{2RT_{int}}}\\
            &&\frac{T_v-T_{int}}{T_{int}}=-0.44675\frac{(u_v-u_{int})\cdot n}{\sqrt{2RT_{int}}}
        \end{align}


    \subsection{ハイパーリンク}
    
        著者が希望する場合に限り,参考文献および文中のコンテンツ(文中・図中・表中)において,原稿外のDOIもしくはURLへのハイパーリンクを利用できます.
        ハイパーリンクを使用する部分は, 黒字とし, 下線などを使用しないで下さい.
        学術論文については, DOIのハイパーリンクとしますが,DOIはリンク内にとどめ,参考文献中には,陽に表記しないでください.
        出版社は変更になる可能性があるため,ハイパーリンクしないで下さい.
    
        非学術論文については,URLのハイパーリンクとし,参考文献中にはURLと共に最終閲覧日を記載してください.(例えば,『最終アクセス日:2021年5月22日』や『accessed 2006-02-01』等).\\
    
    
        \section{その他}
    
        最終ページは左右の段落が出来るだけそろうように調整して下さい.
        各ページの下中央付近に\footnote{%
          ただし,このクラスファイルではページ番号は「ながれ」の印刷形態に
          合わせてヘッダに付けるようにしている.
        },任意のフォントでページ番号を付して下さい.
        本書式と著しく異なる原稿は返却します.
    
    
        \section*{謝辞}
    
        「謝辞」は「本文」の後に置いてください.10ポイント明朝体(英数字はTimes)
        あるいはこれに準じるフォントを使用してください.行間は15ポイントとします.
    
        \appendix % 付録の開始
        \small % 付録の本文を 9pt で書く場合. 8pt にしたければ \footnotesize
    
        \section{付録の位置}
    
        「付録」がある場合は「謝辞」と「引用文献」の間に置いてください.
    
        \section{付録の書式}
    
        付録の本文の書式は付録の長さに応じて,8〜9ポイント明朝体(英数字はTimes)ある
        いはこれに準じるフォントを用い,行間は14ポイントとして下さい.付録の見出しは,
        9ポイント\textgt{ゴシック体}(英数字はTimes)
        あるいはこれに準じるフォントを使用して下さい.
        通し番号(\textbf{付録1},\textbf{付録2}, ...)は
        \textbf{Times-Bold}として下さい.

\end{document}